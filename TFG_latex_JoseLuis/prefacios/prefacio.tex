\chapter*{}
%\thispagestyle{empty}
%\cleardoublepage

%\thispagestyle{empty}

\input{portada/portada_2}



\cleardoublepage
\thispagestyle{empty}

\begin{center}
{\large\bfseries Control de drones mediante bioseñales y FPGA Icezum Alhambra }\\
\end{center}
\begin{center}
José Luis Arco López\\
\end{center}

%\vspace{0.7cm}
\noindent{\textbf{Palabras clave}: Bioseñal, EMG, FPGA, dron, control}\\

\vspace{0.7cm}
\noindent{\textbf{Resumen}}\\

El propósito de este proyecto es la investigación, estudio y finalmente la realización de un sistema de adquisición de señales EMG, con el objetivo de poder usar dichas señales para controlar un dron radiocontrol.  Se presenta como una oportunidad de usar las nuevas tecnologías de las FPGAS con un supuesto bastante necesario a nuestros días: el procesamiento de bioseñales para hacer la vida más fácil. Se comienza estudiando las señales electromiográficas y sus formas de obtención. Para ello se realiza un estudio de estas bioseñales, desde cómo obtenerlas hasta la instrumentación adecuada para tal fin. Tras realizar un banco de datos se desarrolla un algoritmo en Matlab para la extracción de la información de dichas señales para controlar luego el robot. A continuación se implementará este algoritmo en la FPGA de software libre Icezum Alhambra. Por último se realizarán las pruebas necesarias para depurar el sistema con simuladores para luego su puesta en marcha con robots reales.

\cleardoublepage


\thispagestyle{empty}


\begin{center}
{\large\bfseries 
Drone control using biosignals and FPGA Icezum Alhambra}\\
\end{center}
\begin{center}
José Luis Arco López\\
\end{center}

%\vspace{0.7cm}
\noindent{\textbf{Keywords}: biosignal, EMG, FPGA, drone, control}\\

\vspace{0.7cm}
\noindent{\textbf{Abstract}}\\


The purpose of this project is the research, study and finally the realization of an EMG signal acquisition system, with the aim of being able to use these signals to control a radiocontrol drone. It is presented as an opportunity to use the new technologies of the FPGAS with a fairly necessary assumption to this day: the processing of biosignals to make life easier. It begins by studying the electromyographic signals and their ways of obtaining them. For this, a study of these biosignals is carried out, from how to obtain them to the appropriate instrumentation for this purpose. After performing a data bank, an algorithm is developed in Matlab for the extraction of information from these signals to then control the robot. This algorithm will then be implemented in the free software FPGA Icezum Alhambra. Finally, the necessary tests will be carried out to debug the system with simulators and then start up with real robots.
\cleardoublepage

\chapter*{}
\thispagestyle{empty}

\noindent\rule[-1ex]{\textwidth}{2pt}\\[4.5ex]

Yo, \textbf{José Luis Arco López}, alumno de la titulación Ingeniería de Tecnologías de Telecomunicación de la \textbf{Escuela Técnica Superior
de Ingenierías Informática y de Telecomunicación de la Universidad de Granada}, con DNI 77142961H, autorizo la
ubicación de la siguiente copia de mi Trabajo Fin de Grado en la biblioteca del centro para que pueda ser
consultada por las personas que lo deseen.

\vspace{6cm}

\noindent Fdo: José Luis Arco López

\vspace{2cm}

\begin{flushright}
Granada a 11 de mes de 2019.
\end{flushright}


\chapter*{}
\thispagestyle{empty}

\noindent\rule[-1ex]{\textwidth}{2pt}\\[4.5ex]

D.ª \textbf{ María Encarnación del Castillo Morales}, Profesora del Área de Electrónica del Departamento de Electrónica y Tecnología de Computadores de la Universidad de Granada.

\vspace{0.5cm}

D. \textbf{Jose María Cañas Plaza}, Profesor del Área de Telemática del Departamento de Teoría de la Señal, Comunicaciones, Sistemas Telemáticos y Computación de  de la Universidad Rey Juan Carlos de Madrid.


\vspace{0.5cm}

\textbf{Informan:}

\vspace{0.5cm}

Que el presente trabajo, titulado \textit{\textbf{Control de drones mediante bioseñales y FPGA Icezum Alhambra }},
ha sido realizado bajo su supervisión por \textbf{José Luis Arco López}, y autorizamos la defensa de dicho trabajo ante el tribunal
que corresponda.

\vspace{0.5cm}

Y para que conste, expiden y firman el presente informe en Granada a X de mes 11 de 2019 .

\vspace{1cm}

\textbf{Los directores:}

\vspace{5cm}

\noindent \textbf{María Encarnación del Castillo Morales \ \ \ \ \ Jose María Cañas Plaza}

\chapter*{Agradecimientos}
\thispagestyle{empty}

       \vspace{1cm}




